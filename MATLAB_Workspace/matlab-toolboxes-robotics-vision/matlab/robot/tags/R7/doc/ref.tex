\documentclass{article}
%\documentstyle[titlepage,ref,twoside]{article}
\usepackage{verbatim}
\usepackage{fancyheadings}


\setlength{\topmargin}{10.4mm}      % top header - top paper
\setlength{\textheight}{220mm} % height of body, not head and footer
\setlength{\textwidth}{128mm}  % width of text
\setlength{\evensidemargin}{36.6mm} % even page - left paper edge to text
\setlength{\oddsidemargin}{36.6mm}  % odd page - left paper edge to text 
\setlength{\headsep}{12mm} %distance bet. header & body text
\setlength{\footskip}{12mm}%distance bet. body of text & baseline of footer
\setlength{\parindent}{0pt} %don't indent paragraphs
\setlength{\parskip}{2mm} %distance between paragraphs
\setlength{\hfuzz}{10pt} %allow for slight overflow to right margin

% Include font file: either NFSS or PostScript:


%% Fonts defined with long PostScript names 

\newfont{\Rfon}{NewCenturySchlbk-Roman} % Main text font
\newfont{\School}{NewCenturySchlbk-Roman}
\newfont{\si}{NewCenturySchlbk-Italic}
\newfont{\ital}{NewCenturySchlbk-Italic}
\newfont{\emph}{NewCenturySchlbk-Italic}
\newfont{\bs}{NewCenturySchlbk-Bold}

% Courier if you have it, otherwise
% substitute Courier: 
\newfont{\vtt}{Courier at 9pt}
\newfont{\Cour}{Courier at 10pt}
\newfont{\Mono}{Courier at 10pt}
\newfont{\Cfon}{Courier at 9pt}  % Computer font
\newfont{\TTfon}{Courier at 9pt}  % Typewriter font for matlab command display
\newfont{\Lfon}{Courier at 7pt}
\newfont{\Zfon}{Courier at 7pt}

\newfont{\Mfon}{Helvetica-Bold at 14pt}
\newfont{\Refon}{Helvetica-Bold at 12pt} % Font for ref pgs headings
\newfont{\Ifon}{NewCenturySchlbk-Bold scaled \magstep2}
\newfont{\Tfon}{NewCenturySchlbk-Bold scaled \magstep4}
\newfont{\Nfon}{NewCenturySchlbk-Bold at 35pt}
\newfont{\NNfon}{Times-Bold at 32pt}
\newfont{\NNNfon}{Times-Bold at 40pt}
\newfont{\NNNNfon}{NewCenturySchlbk-Bold at 40pt}
\newfont{\rbig}{Times-Bold at 27pt} %vphantom in underlining section heads
\newfont{\tms}{Times-Bold at 18pt} %for section heads
\newfont{\tmss}{Helvetica-Bold at 14pt} %for subsection heads
\newfont{\tmsss}{Helvetica-Bold at 12pt} %for subsubsection heads
\newfont{\tmssss}{Helvetica-BoldOblique at 10pt} %for subsubsubsection heads
\newfont{\nineTR}{NewCenturySchlbk-Roman at 9pt}
\newfont{\nineTI}{NewCenturySchlbk-Italic at 9pt}
\newfont{\elevenTI}{NewCenturySchlbk-Italic at 11pt}
\newfont{\bt}{NewCenturySchlbk-Bold at 9pt}
\newfont{\eightTR}{NewCenturySchlbk-Roman at 8pt}
\newfont{\sevenTR}{NewCenturySchlbk-Roman at 7pt}
\newfont{\eightTI}{NewCenturySchlbk-Italic at 8pt}
\newfont{\Heading}{NewCenturySchlbk-Roman at 70pt}
\newfont{\Headingi}{NewCenturySchlbk-Italic at 60pt}
\newfont{\Afon}{NewCenturySchlbk-Roman at 18pt}

\def\und{{\vtt\char"5F}}


% Fonts defined in NFSS 

%\newfont{\Rfon}{pncr} % Main text font
\newfont{\School}{pncr}
\newfont{\si}{pncri}
%\newfont{\emph}{pncri}
\newfont{\ital}{pncri}
\newfont{\bs}{pncb}

% MathWorks uses Monospace--this
% NFSS definition is for Courier. 
\newfont{\Cour}{pcrr at 10pt}
\newfont{\Mono}{pcrr at 10pt}
\newfont{\vtt}{pcrr at 9pt}
\newfont{\Cfon}{pcrr at 9pt}  % Computer font
\newfont{\TTfon}{pcrr at 9pt}  % Typewriter font for matlab command display
\newfont{\Lfon}{pcrr at 7pt}
\newfont{\Zfon}{pcrr at 7pt}

\newfont{\Mfon}{phvb at 14pt}
\newfont{\Refon}{phvb at 12pt} % Font for ref pgs headings
\newfont{\Ifon}{pncb scaled \magstep2}
\newfont{\Tfon}{pncb scaled \magstep4}
\newfont{\Nfon}{pncb at 35pt}
\newfont{\NNfon}{ptmb at 32pt}
\newfont{\NNNfon}{ptmb at 40pt}
\newfont{\NNNNfon}{pncb at 40pt}
\newfont{\rbig}{ptmb at 27pt} %just for vphantom in underlining section heads
\newfont{\tms}{ptmb at 18pt} %for section heads
\newfont{\tmss}{phvb at 14pt} %for subsection heads
\newfont{\tmsss}{phvb at 12pt} %for subsubsection heads
\newfont{\tmssss}{phvbo at 10pt} %for subsubsubsection heads
\newfont{\nineTR}{pncr at 9pt}
\newfont{\nineTI}{pncri at 9pt}
\newfont{\elevenTI}{pncri at 11pt}
\newfont{\bt}{pncb at 9pt}
\newfont{\eightTR}{pncr at 8pt}
\newfont{\sevenTR}{pncr at 7pt}
\newfont{\eightTI}{pncri at 8pt}
%\newfont{\Heading}{pncb at 70pt}
%\newfont{\Headingi}{pncri at 60pt}
\newfont{\Heading}{ptmb at 70pt}
\newfont{\Headingi}{ptmri at 60pt}
%\newfont{\Afon}{pncr at 18pt}
\newfont{\Afon}{ptmr at 18pt}

\def\und{{\vtt\char"5F}}


%\renewcommand{\rmdefault}{phv}
%\renewcommand{\sfdefault}{phv}
%\renewcommand{\ttdefault}{pcr}

% Define reference page commands.
%
%	\mdes	Description
%	\msa	See also
%	\mex	Examples
%	\mlim	Limitations
%	\mdia	Diagnostics
%	\msyn	Synopsis
%	\mpur	Purpose
%	\malg	Algorithm
%	\mcau	Caution
%	\mref	References
%
\newcommand{\mdes}[1]{\hskip -1.25in 
{\hbox{\makebox[1.25in][l]{\Refon Description}}}{#1}\vskip 0.25in}

\newcommand{\msa}[1]{\hskip -1.25in 
{\hbox{\makebox[1.25in][l]{\Refon See Also}}}{#1}\vskip 0.25in}

\newcommand{\mex}[1]{\hskip -1.25in 
{\hbox{\makebox[1.25in][l]{\Refon Examples}}}{#1}\vskip 0.25in}

\newcommand{\mlim}[1]{\hskip -1.25in 
{\hbox{\makebox[1.25in][l]{\Refon Limitations}}}{#1}\vskip 0.25in}

\newcommand{\mdia}[1]{\hskip -1.25in 
{\hbox{\makebox[1.25in][l]{\Refon Diagnostics}}}{#1}\vskip 0.25in}

\newcommand{\msyn}[1]{\hskip -1.25in 
{\hbox{\makebox[1.25in][l]{\Refon Synopsis}}}{\parbox[t]{10cm}{\Cfon #1}}\vskip 0.25in}

\newcommand{\mpur}[1]{\hskip -1.25in 
{\hbox{\makebox[1.25in][l]{\Refon Purpose}}}{#1}\vskip 0.25in}

\newcommand{\malg}[1]{\hskip -1.25in 
{\hbox{\makebox[1.25in][l]{\Refon Algorithm}}}{#1}\vskip 0.25in}

\newcommand{\mcau}[1]{\hskip -1.25in 
{\hbox{\makebox[1.25in][l]{\Refon Cautionary}}}{#1}\vskip 0.25in}

\newcommand{\mref}[1]{\hskip -1.25in 
{\hbox{\makebox[1.25in][l]{\Refon References}}}{#1}\vskip 0.25in}

\newcommand{\example}[1]{\hskip -1.25in 
{\hbox{\makebox[1.25in][l]{\Refon Example}}}{#1}\vskip 0.25in}

\newcommand{\mauthor}[1]{\hskip -1.25in 
{\hbox{\makebox[1.25in][l]{\Refon Author}}}{#1}\vskip 0.25in}


\newcommand{\function}[1]{\section*{\kern -1.25in\tms #1\hfill\markboth{#1}{#1}}
\vspace*{-4mm}
\moveleft 1.25in\vbox{\makebox[6.25in][l]{\vrule height .01 in width 6.25 true in}}
\vskip .25 true in}

\newcommand{\vvar}[1]{{\vtt #1}}


\newcommand{\var}[1]{{\vtt #1}}
%\newcommand{\newblock}{}

\newcommand{\namelistlabel}[1]{\mbox{#1}\hfil}
\newenvironment{namelist}[1]{%
\begin{list}{}
{
\let\makelabel\namelistlabel
\settowidth{\labelwidth}{#1}
\setlength{\leftmargin}{1.1\labelwidth}
}
}{%
\end{list}}

\renewcommand{\hom}{homogeneous transform}
\newcommand{\Hom}{Homogeneous transform}
\renewcommand{\vec}[1]{\underline{#1}}
\input{psfig}

\begin{document}


\newcommand{\Mlab}{{\School M}{\eightTR ATLAB}}
\newcommand{\MLAB}{{\emph M}{\eightIT ATLAB}}
\newcommand{\undersec}[1]{\underline{\rule[-.70ex]{0cm}{0cm}#1}}


%{\School
%\pagestyle{headings}        % Gives page headings at top of page
\pagestyle{fancyplain}
\rhead{\rm\thepage}
\lfoot{Robotics Toolbox Release 7}
\rfoot{Peter Corke, April 2002}
\cfoot{}
\setlength{\headrulewidth}{0pt}

\newpage   % Starts introduction on RH page
\def\be{\begin{equation}}
\def\ee{\end{equation}}
\newcommand{\under}[1]{\underline{\rule[-.70ex]{0cm}{0cm}#1}}
\renewcommand{\baselinestretch}{1.2}           % Stretches line spacing
\tolerance=10000


\rightline{\NNNfon 2}
\vskip 4mm
\rightline{\NNfon Reference}
\vskip 2mm
\moveleft 1.25in\vbox{\makebox[6.25in][l]
{\vrule height .01 in width 6.25 true in}}

\vskip 5mm
%\vfil\eject 
\addcontentsline{toc}{section}{Introduction}

For an n-axis manipulator the following matrix naming and dimensional
conventions apply.

\begin{tabular}
{|lp{30mm}p{70mm}|}\hline
Symbol & Dimensions & Description\\ \hline
\var{l} & \textbf{link} & manipulator link object\\
\var{q} & $1 \times n$ & joint coordinate vector\\
\var{q} & $m \times n$ & $m$-point joint coordinate trajectory \\
\var{qd} & $1 \times n$ & joint velocity vector\\
\var{qd} & $m \times n$ & $m$-point joint velocity trajectory \\
\var{qdd} & $1 \times n$ & joint acceleration vector\\
\var{qdd} & $m \times n$ & $m$-point joint acceleration trajectory \\
\var{robot} & \textbf{robot} & robot object\\
\var{T} & $4 \times 4$ & \hom \\
\var{T} & $4 \times 4 \times m$ & $m$-point \hom\ trajectory\\
\var{Q} & \textbf{quaternion} & unit-quaternion object\\
\var{M} & $1 \times 6$ & vector with elements of 0 or 1 corresponding
to Cartesian DOF along X, Y, Z and around X, Y, Z.  1 if that Cartesian
DOF belongs to the task space, else 0.\\
\var{v} & $3 \times 1$ & Cartesian vector\\
\var{t} & $m\times 1$ & time vector\\
\var{d} & $6\times 1$ & differential motion vector\\ \hline
\end{tabular}

Object names are shown in bold typeface.

A trajectory is represented by a matrix in which each row corresponds to
one of $m$ time steps.
For a joint coordinate, velocity or acceleration trajectory the columns correspond
to the robot axes.
For \hom\ trajectories we use 3-dimensional matrices where the last
index corresponds to the time step.

\section*{Units}
All angles are in radians.
The choice of all other units is up to the user, and this choice will flow
on to the units in which \hom s, Jacobians, inertias and torques
are represented.

\vfil\eject
\markboth{Introduction}{Introduction}

%\begin{tabular}
%{|p{1.25in}p{3.25in}|}\cline{1-2}
%&\\
%\multicolumn{2}{|c|}{\tmsss Reference Page Commands}\\
%\hline
%&\\
%{\Mono ml\_demo1,clc} & User-written function to convert parameter
%vector into a state-space system model.\\
%{\Mono wersum} & Variable augmented with a-priori weighting term.\\
%&\\
%\cline{1-2}
%\end{tabular}

%%SUMMARY_START
\begin{tabular}
{|p{1.25in}p{3.25in}|}\cline{1-2}
&\\
\multicolumn{2}{|c|}{\tmsss Homogeneous Transforms} \\ \hline
{\Mono eul2tr} & Euler angle to \hom \\
{\Mono oa2tr} & orientation and approach vector to \hom \\
{\Mono rotvec} & \hom\ for rotation about arbitrary vector \\
{\Mono rotx} & \hom\ for rotation about X-axis \\
{\Mono roty} & \hom\ for rotation about Y-axis \\
{\Mono rotz} & \hom\ for rotation about Z-axis \\
{\Mono rpy2tr} & Roll/pitch/yaw angles to \hom \\
{\Mono tr2eul} & \hom\ to Euler angles \\
{\Mono tr2rot} & \hom\ to rotation submatrix\\
{\Mono tr2rpy} & \hom\ to roll/pitch/yaw angles\\ 
{\Mono transl} & set or extract the translational component of a \hom \\
{\Mono trnorm} & normalize a \hom \\ \hline
\end{tabular}

\begin{tabular}
{|p{1.25in}p{3.25in}|}\cline{1-2}
&\\
\multicolumn{2}{|c|}{\tmsss Trajectory Generation} \\ \hline
{\Mono ctraj} & Cartesian trajectory \\
%{\Mono drivepar} & Cartesian trajectory parameters \\
{\Mono jtraj} & joint space trajectory \\ 
{\Mono trinterp} & interpolate \hom s\\  \hline
\end{tabular}


\begin{tabular}
{|p{1.25in}p{3.25in}|}\cline{1-2}
&\\
\multicolumn{2}{|c|}{\tmsss Quaternions} \\ \hline
{\Mono /} & divide quaternion by quaternion or scalar\\
{\Mono *} & multiply quaternion by a quaternion or vector\\
{\Mono inv} & invert a quaternion \\
{\Mono norm} & norm of a quaternion \\
{\Mono plot} & display a quaternion as a 3D rotation\\
{\Mono q2tr} & quaternion to \hom\\
{\Mono quaternion} & construct a quaternion \\
{\Mono qinterp} & interpolate quaternions\\ 
{\Mono unit} & unitize a quaternion \\  \hline
\end{tabular}

\begin{tabular}
{|p{1.25in}p{3.25in}|}\cline{1-2}
&\\
\multicolumn{2}{|c|}{\tmsss Manipulator Models} \\ \hline
{\Mono link} & construct a robot link object \\
{\Mono nofriction} & remove friction from a robot object \\
{\Mono perturb} & randomly modify some dynamic parameters\\
{\Mono puma560} & Puma 560 data \\
{\Mono puma560akb} & Puma 560 data (modified Denavit-Hartenberg)\\
{\Mono robot} & construct a robot object \\ 
{\Mono showlink} & show link/robot data in detail\\ 
{\Mono stanford} & Stanford arm data \\ 
{\Mono twolink} & simple 2-link example \\ \hline
\end{tabular}

\begin{tabular}
{|p{1.25in}p{3.25in}|}\cline{1-2}
&\\
\multicolumn{2}{|c|}{\tmsss Kinematics} \\ \hline
{\Mono diff2tr} & differential motion vector to transform \\
{\Mono fkine} & compute forward kinematics \\
{\Mono ftrans} & transform force/moment \\
{\Mono ikine} & compute inverse kinematics \\
{\Mono ikine560} & compute inverse kinematics for Puma 560 like arm\\
{\Mono jacob0} & compute Jacobian in base coordinate frame\\
{\Mono jacobn} & compute Jacobian in end-effector coordinate frame\\
{\Mono tr2diff} & \hom\ to differential motion vector \\
{\Mono tr2jac} & \hom\ to Jacobian \\ \hline
\end{tabular}

\begin{tabular}
{|p{1.25in}p{3.25in}|}\cline{1-2}
&\\
\multicolumn{2}{|c|}{\tmsss Graphics} \\ \hline
{\Mono drivebot} & drive a graphical  robot \\ 
{\Mono plot} & plot/animate robot \\ \hline
\end{tabular}

\begin{tabular}
{|p{1.25in}p{3.25in}|}\cline{1-2}
&\\
\multicolumn{2}{|c|}{\tmsss Dynamics} \\ \hline
{\Mono accel} & compute forward dynamics\\
{\Mono cinertia} & compute Cartesian manipulator inertia matrix \\
{\Mono coriolis} & compute centripetal/coriolis torque \\
{\Mono fdyn} & forward dynamics (motion given forces) \\
{\Mono friction} & joint friction\\
{\Mono gravload} & compute gravity loading \\
{\Mono inertia} & compute manipulator inertia matrix \\
{\Mono itorque} & compute inertia torque \\
{\Mono rne} & inverse dynamics (forces given motion)\\ \hline
\end{tabular}

\begin{tabular}
{|p{1.25in}p{3.25in}|}\cline{1-2}
&\\
\multicolumn{2}{|c|}{\tmsss Other} \\ \hline
%{\Mono cross} & vector cross product \\
{\Mono ishomog} & test if matrix is $4 \times 4$\\
{\Mono maniplty} & compute manipulability \\
{\Mono rtdemo} & toolbox demonstration\\
{\Mono unit} & unitize a vector\\ \hline
\end{tabular}
%%SUMMARY_END
\vfil\eject


%\section*{\kern -1.25in\tms mmle.m\hfill\markboth{mmle.m}{mmle.m}}
%\vspace*{-4mm}
%\moveleft 1.25in\vbox{\makebox[6.25in][l]{\vrule height .01 in width 6.25 true in}}
%\vskip .25 true in
%\mpur{Handle models with known or unknown measurement and/or state noise.}
%
%\synopsis{\Cfon
%{\obeylines
%format compact
%
%load ml\_demo1,clc}
%} 
%
%\mdes{A simple user-written function is created to convert
%the parameter vector into a state-space system model, some input
%options are defined and {\vtt mmle.m} executed.}
%
%\mex{Use standard {\Mlab} \ commands to plot any of the above inputs
%and outputs.
%The innovations can be plotted using :
%\begin{center}
%{\vtt plot(dt$\times$[0:ndp-1], uydata(:,ym)-yest) }
%\end{center}}
%
%\malg{The variable, {\vtt wersum} is
%augmented with the a-priori weighting term to give, using program variable
%names, 
%\be \mbox{wersum}
% =
%\frac{1}{Nm} \left[ \sum_{i=1}^N \mbox{inov}^T \mbox{gg}^{-1} \mbox{inov} + 
%(\mbox{p} - \mbox{pref})^T \mbox{wapriori} (\mbox{p} - \mbox{pref})  \right] 
%\ee
%
%Variable {\vtt inov} is the innovations sequence from the Kalman filter,
%{\vtt gg} is the assumed innovations covariance,
%{\vtt p} is the parameter vector,
%{\vtt pref} is the reference value, and
%{\vtt wapriori} is the a-priori weighting.
%The a-priori weighting can be used to bias the estimated parameters
%in a (possibly wrong) direction favored by the analyst.
%
%The Maximum Likelihood Estimate at the minimum of the LLF  (Log-Likelihood Funct
%ion) is achieved by making
%{\vtt wapriori}  zero and by using the correct value for {\vtt gg}, i.e., \be
%\mbox{gg} \Leftarrow \mbox{rr} \equiv \frac{1}{N} \sum_{i=1}^{N}
%\mbox{\ inov} \mbox{\ inov}^T \label{samcov}
%\ee}
%
%\mcau{Beware of special cases.}
%
%\mlim{Assume a linear time-independent noise model.} 
%
%\msa{\Cfon ml\_p2ss3.m}
%
%\mdia{At each parameter optimization step {\vtt mmle.m} prints :
%\begin{center}
%\begin{tabular}{lcl}
% {\vtt STEP}       &  :& step number \\
%{\vtt PIDF\und nsum}   & :& sum of new {\vtt p(pidf)}, where  {\vtt p} = parameter vector \\
% {\vtt GG\und osum}     & :& old sum of innovations cov. used during the step \\
% {\vtt RR\und nsum}     & :& sum of innovations cov. with new parameters \\
% {\vtt MAXGRAD}    &  :& max(abs(grad)) for {\si old} parameter values \\
% {\vtt MAX$|$dP$|$}   &   :& largest parameter change \\
% {\vtt LLF}      &    :& LLF with new parameter values \\
% {\vtt WERSUM}  &   :& See the {\si Tutorial} and Equation 1. \\
% {\vtt ppid}    &    : & {\vtt p(pid)} = latest identified parameter values \\
%\end{tabular}
%\end{center}}
%
%\mref{Information on references is in the {\si Tutorial}.} 
%
%\vfil\eject 
%

%%%%%%%%%%%%%%%%%%%%%%%%%%%%%%%%%%%%%%%
\function{accel}
\mpur{Compute manipulator forward dynamics}

\msyn{
qdd = accel(robot, q, qd, torque)
}
\mdes{Returns a vector of joint accelerations
that result from applying the actuator \var{torque} to the manipulator 
\var{robot} with joint coordinates \var{q} and velocities \var{qd}.
}

\malg{Uses the method 1 of Walker and Orin to compute the forward dynamics.  This
form is useful for simulation of manipulator dynamics, in conjunction with
a numerical integration function.}

\msa{rne, robot, fdyn, ode45}
\mref{M.~W. Walker and D.~E. Orin.
\newblock Efficient dynamic computer simulation of robotic mechanisms.
\newblock {\em ASME Journal of Dynamic Systems, Measurement and Control},
  104:205--211, 1982.}

\vfil\eject 


%%%%%%%%%%%%%%%%%%%%%%%%%%%%%%%%%%%%%%%
\function{cinertia}
\mpur{
Compute the Cartesian (operational space) manipulator inertia matrix}

\msyn{
M = cinertia(robot, q)
}

\mdes{\var{cinertia} computes the Cartesian, or operational space, inertia matrix.
\var{robot} is a robot object that describes the manipulator dynamics 
and kinematics, and \var{q} is
an n-element vector of joint coordinates.}


\malg{The Cartesian inertia matrix is calculated from the joint-space inertia
matrix by
\[
{\bf M}(\vec{x}) = {\bf J}(\vec{q})^{-T} {\bf M}(\vec{q}) {\bf J}(\vec{q})^{-1}
\]
and relates Cartesian force/torque to Cartesian
acceleration
\[
\vec{F} = {\bf M}(\vec{x}) \vec{\ddot{x}}
\]
}
\msa{inertia, robot, rne}
\mref{O.~Khatib, ``A unified approach for motion and force control of robot
  manipulators: the operational space formulation,'' {\em IEEE Trans. Robot.
    Autom.}, vol.~3, pp.~43--53, Feb. 1987.}
\vfil\eject 


%%%%%%%%%%%%%%%%%%%%%%%%%%%%%%%%%%%%%%%
\function{coriolis}
\mpur{Compute the manipulator Coriolis/centripetal torque components}

\msyn{
tau\_c = coriolis(robot, q, qd)
}
\mdes{\var{coriolis} returns the joint torques due to rigid-body
Coriolis and centripetal effects for the specified joint state
\var{q} and velocity \var{qd}.
\var{robot} is a robot object that 
describes the manipulator dynamics and kinematics.

If \var{q} and \var{qd} are row vectors, \var{tau\_c} is a row vector 
of joint torques.
If \var{q} and \var{qd} are matrices, each row is interpreted as a joint state 
vector, and \var{tau\_c} is a matrix each row being the 
corresponding joint torques.
}


\malg{Evaluated from the equations of motion, using \var{rne}, with
joint acceleration and gravitational acceleration set to zero,
\[
\vec{\tau} = {\bf C}(\vec{q}, \vec{\dot{q}}) \vec{\dot{q}}
\]

Joint friction is ignored in this calculation.}

\msa{robot, rne, itorque, gravload}
\mref{M.~W. Walker and D.~E. Orin.
\newblock Efficient dynamic computer simulation of robotic mechanisms.
\newblock {\em ASME Journal of Dynamic Systems, Measurement and Control},
  104:205--211, 1982.}

\vfil\eject 

%%%%%%%%%%%%%%%%%%%%%%%%%%%%%%%%%%%%%%%
%\function{cross}
%\mpur{Vector cross product}
%
%\msyn{
%v = cross(v1, v2)
%}
%\mdes{Returns the vector cross product \var{v1} $\times$ \var{v2}
%The result has the same orientation (row or column vector) as the two
%inputs.
%
%Note that this function is no longer provided with the Toolbox since it is
%now part of the core MATLAB toolbox (\var{specfun/cross.m}).}
%
%\vfil\eject 
%
%%%%%%%%%%%%%%%%%%%%%%%%%%%%%%%%%%%%%%%
\function{ctraj}
\mpur{Compute a Cartesian trajectory between two points}

\msyn{
TC = ctraj(T0, T1, m)\\[0pt]
TC = ctraj(T0, T1, r)
}
\mdes{\var{ctraj} returns a Cartesian trajectory (straight line motion)
\var{TC} from the point
represented by \hom\ \var{T0} to \var{T1}.  
The number
of points along the path is \var{m} or the length of the given  vector \var{r}.
For the second case \var{r} is a vector of distances along the path (in the
range 0 to 1) for
each point.  The first case has the points equally spaced, but different
spacing may be specified to achieve acceptable acceleration profile.
\var{TC} is a $4 \times 4 \times m$ matrix.
}

\mex{To create a Cartesian path with smooth acceleration we can use the
\var{jtraj} function to create the path vector \var{r} with continuous
derivitives.}
\begin{verbatim}
>> T0 = transl([0 0 0]);  T1 = transl([-1 2 1]);
>> t= [0:0.056:10];
>> r = jtraj(0, 1, t);
>> TC = ctraj(T0, T1, r);
>> plot(t, transl(TC));
\end{verbatim}

\centerline{\psfig{figure=figs/ctraj.eps,width=8cm}}

\msa{trinterp, qinterp, transl}
\mref{R.~P. Paul, {\em Robot Manipulators: Mathematics, Programming, and Control}.
\newblock Cambridge, Massachusetts: MIT Press, 1981.}
\vfil\eject 


%%%%%%%%%%%%%%%%%%%%%%%%%%%%%%%%%%%%%%%
\function{diff2tr}
\mpur{Convert a differential motion vector to a homogeneous transform}

\msyn{
delta = diff2tr(x)
}

\mdes{Returns a homogeneous transform representing differential
translation and rotation corresponding to Cartesian velocity
$x = [v_x\; v_y\; v_z\; \omega_x\; \omega_y\; \omega_z]$.}

%\malg{For a differential motion $\Delta = [d_x\; d_y\; d_z\; \delta_x\; \delta_y\; \delta_z]$ the corresponding \hom\ is
%\[
%\Delta = \left[ \begin{array}{cccc}
%	0 &	 -\delta_z &	 \delta_y &	d_x \\
%	\delta_z &	0 &	-\delta_x &	d_y \\
%	-\delta_y &	\delta_x & 0 &		d_z \\
%	0 &		0 &	0 &		0
%	\end{array} \right]
%\]
%Note that the rotational submatrix is skew-symmetric.}
\malg{From mechanics we know that
\[
\dot{\bf R} = Sk(\Omega) {\bf R}
\]
where $\bf R$ is an orthonormal rotation matrix and
\[
{\bf Sk}(\Omega) = \left[ \begin{array}{ccc}
	0 &	 -\omega_z &	 \omega_y \\
	\omega_z &	0 &	-\omega_x \\
	-\omega_y &	\omega_x & 0 \end{array} \right]
\]
and is a skew-symmetric matrix.
This can be generalized to
\[
\dot{\bf T} = \left[ \begin{array}{cc}
	Sk(\Omega) & \dot{P} \\
	0 0 0 & 1 \end{array} \right] {\bf T}
\]
for the rotational and translational case.}

\msa{tr2diff}
\mref{R.~P. Paul.
\newblock {\em Robot Manipulators: Mathematics, Programming, and Control}.
\newblock MIT Press, Cambridge, Massachusetts, 1981.}
\vfil\eject 


%%%%%%%%%%%%%%%%%%%%%%%%%%%%%%%%%%%%%%%
\function{drivebot}
\mpur{Drive a graphical robot}

\msyn{
drivebot(robot)\\[0pt]
drivebot(robot, q)
}

\centerline{\psfig{figure=figs/drivebot.eps,width=8cm}}

\mdes{Pops up a window with one slider for each joint.  Operation
of the sliders will drive the graphical robot on the screen.
Very useful for gaining an understanding of joint limits and robot
workspace.

The joint coordinate state is kept with the graphical robot and can
be obtained using the \var{plot} function.
If \var{q} is specified it is used as the initial joint angle, otherwise
the initial value of
joint coordinates is taken from the graphical robot.
}

\mex{To drive a graphical Puma 560 robot}
\begin{verbatim}
>> puma560           % define the robot
>> plot(p560,qz)     % draw it
>> drivebot(p560)    % now drive it
\end{verbatim}

\msa{robot/plot,robot}

\vfil\eject 
%%%%%%%%%%%%%%%%%%%%%%%%%%%%%%%%%%%%%%%%
%\function{drivepar}
%\mpur{Compute Cartesian path drive parameters}
%
%\msyn{
%dp = drivepar(T0, T1)
%}
%
%\mdes{\var{drivepar} returns a vector which represents the `difference'
%between \hom\ \var{T0} and \var{T1} as a translation and three angles.
%This is most frequently used in planning a Cartesian path between 
%\hom\ \var{T0} and \var{T1} and \var{dp} would be passed  to \var{trinterp}.
%}
%
%\msa{ctraj, trinterp}
%\mref{R.~P. Paul, {\em Robot Manipulators: Mathematics, Programming, and Control}.
%\newblock Cambridge, Massachusetts: MIT Press, 1981.}
%\vfil\eject 
%
%
%%%%%%%%%%%%%%%%%%%%%%%%%%%%%%%%%%%%%%%
\function{eul2tr}
\mpur{Convert Euler angles to a \hom}

\msyn{
T = eul2tr([r p y])\\[0pt]
T = eul2tr(r,p,y)
}

\mdes{\var{eul2tr} returns a homogeneous transformation for the specified
Euler angles in radians.
\[
T = R_Z(a) R_Y(b) R_Z(c)
\]
}

\mcau{Note that 12 different Euler angle sets or conventions exist.  The
convention used here is the common one for robotics, but is not the one used
for example in the aerospace community.}

\msa{tr2eul, rpy2tr}
\mref{R.~P. Paul, {\em Robot Manipulators: Mathematics, Programming, and Control}.
\newblock Cambridge, Massachusetts: MIT Press, 1981.}
\vfil\eject 

%%%%%%%%%%%%%%%%%%%%%%%%%%%%%%%%%%%%%%%
\function{fdyn}
\mpur{Integrate forward dynamics}

\msyn{
[t q qd] = fdyn(robot, t0, t1)\\[0pt]
[t q qd] = fdyn(robot, t0, t1, torqfun)\\[0pt]
[t q qd] = fdyn(robot, t0, t1, torqfun, q0, qd0)
[t q qd] = fdyn(robot, t0, t1, torqfun, q0, qd0, arg1, arg2, ...)
}

\mdes{\var{fdyn} integrates the manipulator equations of motion over the time interval 
\var{t0} to \var{t1} using \Mlab's \var{ode45} numerical integration function.
Manipulator kinematic and dynamic chacteristics are given by the robot
object \var{robot}.
It returns a time vector \var{t}, and matrices of manipulator joint state \var{q} and
joint velocities \var{qd}.
These matrices have one row per time step and one column per joint.

Actuator torque may be specified by a user function
\begin{center}
\var{tau = torqfun(t, q, qd, arg1, arg2, ...)}
\end{center}

where \var{t} is the current time, and 
\var{q} and \var{qd} are the manipulator joint coordinate
and velocity state respectively.
Optional arguments passed to \var{fdyn} will be passed through to the 
user function.
Typically this function would be used to implement some axis control scheme as a
function of manipulator state and passed in setpoint information.
If \var{torqfun} is not specified then zero torque is applied to the 
manipulator.

Initial joint coordinates and velocities may be specified by the optional
arguments \var{q0} and \var{qd0} respectively.
}

\malg{The joint acceleration is a function of joint coordinate and velocity given by
\[
\vec{\ddot{q}} = {\bf M}(\vec{q})^{-1} \left\{ \vec{\tau} - {\bf C}(\vec{q}, \vec{\dot{q}}) \vec{\dot{q}} - {\bf G}(\vec{q}) - {\bf F}(\vec{\dot{q}}) \right\}
\]}

\example{The following example shows how \var{fdyn()} can be used to simulate a
robot and its controller.  The manipulator is a Puma 560 with simple
proportional and  derivative controller.  The simulation results are shown in
the figure, and further gain tuning is clearly required.  Note that high gains
are required on joints 2 and 3 in order to counter the significant disturbance
torque due
to gravity.
}

{\small
\begin{verbatim}
    >> puma560                         % load Puma parameters
    >> t = [0:.056:5]';                % time vector
    >> q_dmd = jtraj(qz, qr,t);        % create a path
    >> qt = [t q_dmd];
    >> Pgain = [20 100 20 5 5 5];   % set gains
    >> Dgain = [-5 -10 -2 0 0 0];
    >> [tsim,q,qd] = fdyn(nofriction(p560), 0, 5, 'taufunc', qz, qz, Pgain, Dgain, qt);
\end{verbatim}}
\noindent
Note the use of \var{qz} a zero vector of length 6 defined by \var{puma560}
pads out the two initial condition arguments, and we place the control
gains and the path as optional arguments.
Note also the use of the \var{nofriction()} function, see Cautionary note below.
The invoked function is
{\small
\begin{verbatim}
    %
    %        taufunc.m
    %
    % user written function to compute joint torque as a function
    % of joint error.  The desired path is passed in via the global
    % matrix qt.  The controller implemented is PD with the proportional
    % and derivative gains given by the global variables Pgain and Dgain
    % respectively.
    %
function tau = taufunc(t, q, qd, Pgain, Dgain, qt)

        % interpolate demanded angles for this time
	if t > qt(end,1),        % keep time in range
		t = qt(end,1);
	end
	q_dmd = interp1(qt(:,1), qt(:,2:7), t)';

	% compute error and joint torque
	e = q_dmd - q;
	tau = diag(Pgain)*e + diag(Dgain)*qd;
\end{verbatim}}

{\centerline{\psfig{figure=figs/taufunc.eps,height=8cm}}
Results of \var{fdyn()} example.  Simulated path shown as
solid, and reference path as dashed.}

\mcau{The presence of non-linear friction in the dynamic model can prevent the
integration from converging.  The function \var{nofriction()} can be
used to return a Coulomb friction free robot object.}
\msa{accel, nofriction, rne, robot, ode45}
\mref{M.~W. Walker and D.~E. Orin.
\newblock Efficient dynamic computer simulation of robotic mechanisms.
\newblock {\em ASME Journal of Dynamic Systems, Measurement and Control},
  104:205--211, 1982.
}

\vfil\eject 

%%%%%%%%%%%%%%%%%%%%%%%%%%%%%%%%%%%%%%%
\function{fkine}
\mpur{
Forward robot kinematics for serial link manipulator}

\msyn{
T = fkine(robot, q)
}

\mdes{\var{fkine}
computes forward kinematics for the joint coordinate \var{q} giving a 
\hom\ for the location of the end-effector.
\var{robot} is a robot object which contains a kinematic model in either
standard or modified Denavit-Hartenberg notation.
Note that the robot object can specify an arbitrary \hom\ for the base of
the robot and a tool offset.

If \var{q} is a vector it is interpreted as the generalized joint coordinates,
and \var{fkine} returns a homogeneous transformation for the final link of the
manipulator. If \var{q} is a matrix each row is interpreted as a joint state
vector, and \var{T} is a $4 \times 4 \times m$ matrix where \var{m} is the
number  of rows in \var{q}.
}
\mcau{
Note that the dimensional units for the last column of the \var{T} matrix
will be the same as the dimensional units used in the \var{robot} object.  The
units can be whatever you choose (metres, inches,
cubits or furlongs) but the choice will affect the numerical value of the 
elements in the last column of \var{T}.
The Toolbox definitions \var{puma560} and \var{stanford}
all use SI units with dimensions in metres.}

\msa{link, robot}
\mref{R.~P. Paul.
\newblock {\em Robot Manipulators: Mathematics, Programming, and Control}.
\newblock MIT Press, Cambridge, Massachusetts, 1981.\\
J.~J. Craig, {\em Introduction to Robotics}.
\newblock Addison Wesley, second~ed., 1989.}

\vfil\eject 

%%%%%%%%%%%%%%%%%%%%%%%%%%%%%%%%%%%%%%%
\function{link/friction}
\mpur{Compute joint friction torque}

\msyn{
tau\_f = friction(link, qd)
}

\mdes{\var{friction} computes the joint friction torque based on friction
parameter data, if any, in the link object \var{link}.
Friction is a function only of joint velocity \var{qd}

If \var{qd} is a vector then
\var{tau\_f} is a vector in which each element is the friction torque for the
the corresponding element in \var{qd}.
}

\malg{The friction model is a fairly standard one comprising viscous friction
and direction dependent Coulomb friction
\[
F_{i}(t) = \cases{ {B_{i}}\, \dot{q} + {\tau_{i}^{-}}  \; ,& 
			$ \dot{\theta} < 0 $ \cr
		   {B_{i}}\, \dot{q} + {\tau_{i}^{+}}  \; ,& 
			   $ \dot{\theta} > 0 $ \cr}
\]
}
\msa{link,robot/friction,nofriction}

\vfil\eject 
%%%%%%%%%%%%%%%%%%%%%%%%%%%%%%%%%%%%%%%
\function{robot/friction}
\mpur{Compute robot friction torque vector}

\msyn{
tau\_f = friction(robot, qd)
}

\mdes{\var{friction} computes the joint friction torque vector
for the robot object \var{robot} with a joint velocity vector \var{qd}.
}

\msa{link, link/friction, nofriction}

\vfil\eject 
%%%%%%%%%%%%%%%%%%%%%%%%%%%%%%%%%%%%%%%
\function{ftrans}
\mpur{Force transformation}

\msyn{
F2 = ftrans(F, T)
}

\mdes{Transform the force vector \var{F} in the current coordinate
frame to force vector \var{F2} in the second coordinate frame.  The second
frame is related to the first by the \hom\ \var{T}.
\var{F2} and \var{F} are each 6-element vectors comprising force and
moment components $[F_x \; F_y \; F_z \; M_x \; M_y \; M_z]$.
}


\msa{diff2tr}


\vfil\eject 
%%%%%%%%%%%%%%%%%%%%%%%%%%%%%%%%%%%%%%%
\function{gravload}
\mpur{Compute the manipulator gravity torque components}

\msyn{
tau\_g = gravload(robot, q)\\[0pt]
tau\_g = gravload(robot, q, grav)
}

\mdes{\var{gravload} computes the joint torque due to gravity for the 
manipulator in pose \var{q}.

If \var{q} is a row vector, \var{tau\_g} returns a row vector of joint torques.
If \var{q} is a matrix each row is interpreted as as a joint state vector, and
\var{tau\_g} is a matrix in which each row is the gravity torque for the
the corresponding row in \var{q}.

The default gravity direction comes from the robot object but
may be overridden by the optional \var{grav} argument.
}

\msa{robot, link, rne, itorque, coriolis}
\mref{M.~W. Walker and D.~E. Orin.
\newblock Efficient dynamic computer simulation of robotic mechanisms.
\newblock {\em ASME Journal of Dynamic Systems, Measurement and Control},
  104:205--211, 1982.
  }
\vfil\eject 

%%%%%%%%%%%%%%%%%%%%%%%%%%%%%%%%%%%%%%%
\function{ikine}
\mpur{Inverse manipulator kinematics}

\msyn{
q = ikine(robot, T)\\[0pt]
q = ikine(robot, T, q0)\\[0pt]
q = ikine(robot, T, q0, M)
}

\mdes{\var{ikine} returns the joint coordinates for the manipulator
described by the object \var{robot} whose
end-effector \hom\  is given by \var{T}.
Note that the robot's base can be arbitrarily specified within the robot
object.

If \var{T} is a \hom\ then a row vector of joint coordinates is returned.
The estimate for the first step is \var{q0} if this is given else 0.

If \var{T} is a \hom\ trajectory of size $4 \times 4 \times m$ then \var{q}
 will be an $n \times m$ matrix where each row is a vector of joint
 coordinates corresponding to the last subscript of \var{T}. 
The estimate for the first step in the sequence is \var{q0} if this is given 
else 0.
The initial estimate of \var{q} for each time
step is taken as the solution from the previous time step.

Note  that the inverse kinematic solution is
generally not unique, and depends on the initial value \var{q0} (which defaults to 0).

For the case of a manipulator with fewer than 6 DOF it is not possible for
the end-effector to satisfy the end-effector pose specified by an 
arbitrary \hom.  This typically leads to non-convergence in \var{ikine}.
A solution is to specify a 6-element weighting vector, \var{M}, whose
elements are 0 for those Cartesian DOF that are unconstrained and 1 otherwise.
The elements correspond to translation along the X-, Y- and Z-axes and rotation
about the X-, Y- and Z-axes respectively.
For example, a 5-axis manipulator may be incapable of independantly controlling
rotation about the end-effector's Z-axis.  In this case \var{M = [1 1 1 1 1 0]}
would enable a solution in which the end-effector adopted the pose \var{T}
{\it except} for the end-effector rotation.
The number of non-zero elements should equal the number of robot DOF.}

\malg{The solution is computed iteratively using the pseudo-inverse 
of the manipulator Jacobian.
\[
\vec{\dot{q}} = {\bf J}^+(\vec{q}) \Delta \left( {\cal F}(\vec{q}) - T \right)
\]
where $\Delta$ returns the `difference' between two transforms as
a  6-element vector of displacements and rotations (see \var{tr2diff}).}

\mcau{Such a solution is completely general, though much less efficient than specific
inverse kinematic solutions derived symbolically.

The returned joint angles may be in non-minimum form, ie. $q + 2n\pi$.

This approach allows a solution to obtained at a singularity, but the joint 
coordinates within the null space are arbitrarily assigned.

Note that the dimensional units used for the last column of the \var{T} matrix
must agree with the dimensional units used in the robot definition.  These
units can be whatever you choose (metres, inches, cubits or furlongs) but they
must be consistent.  The Toolbox definitions \var{puma560} and \var{stanford}
all use SI units with dimensions in metres.}

\msa{fkine, tr2diff, jacob0, ikine560, robot}

\mref{S.~Chieaverini, L.~Sciavicco, and B.~Siciliano, ``Control of robotic systems
  through singularities,'' in {\em Proc. Int. Workshop on Nonlinear and
    Adaptive Control: Issues in Robotics} (C.~C. de~Wit, ed.), Springer-Verlag,
      1991.}

\vfil\eject 
%%%%%%%%%%%%%%%%%%%%%%%%%%%%%%%%%%%%%%%
\function{ikine560}
\mpur{Inverse manipulator kinematics for Puma 560 like arm}

\msyn{
q = ikine560(robot, config)
}

\mdes{\var{ikine560} returns the joint coordinates corresponding to the
end-effector \hom\  \var{T}.
It is computed using a symbolic solution appropriate for Puma 560 like robots,
that is, all revolute 6DOF arms, with a spherical wrist.
The use of a symbolic solution means that it executes over 50 times
faster than \var{ikine} for a Puma 560 solution.

A further advantage is that \var{ikine560()} allows control over the
specific solution returned.
\var{config} is a string which contains one or more of the configuration
control letter codes

\begin{center}
\begin{tabular}{rl}
\var{'l'} & left-handed (lefty) solution (default)\\
\var{'r'} & $\dagger$right-handed (righty) solution\\
\var{'u'} & $\dagger$elbow up solution (default)\\
\var{'d'} & elbow down solution \\
\var{'f'} & $\dagger$wrist flipped solution \\
\var{'n'} & wrist not flipped solution (default)\\
\end{tabular}
\end{center}
}

\mcau{Note that the dimensional units used for the last column of the \var{T} matrix
must agree with the dimensional units used in the \var{robot} object.  These
units can be whatever you choose (metres, inches, cubits or furlongs) but they
must be consistent.  The Toolbox definitions \var{puma560} and \var{stanford}
all use SI units with dimensions in metres.}

\msa{fkine, ikine, robot}


\mref{R.~P. Paul and H.~Zhang, ``Computationally efficient kinematics for
  manipulators with spherical wrists,'' {\em Int. J. Robot. Res.}, vol.~5,
    no.~2, pp.~32--44, 1986.}

\mauthor{Robert Biro and Gary McMurray, Georgia Institute of Technology, gt2231a@acmex.gatech.edu}
\vfil\eject 


%%%%%%%%%%%%%%%%%%%%%%%%%%%%%%%%%%%%%%%
\function{inertia}
\mpur{Compute the manipulator joint-space inertia matrix}

\msyn{
M = inertia(robot, q)
}

\mdes{\var{inertia} computes the joint-space
inertia matrix which relates joint torque to joint acceleration
\[
\vec{\tau} = {\bf M}(\vec{q}) \vec{\ddot{q}}
\]
\var{robot} is a robot object that describes the manipulator dynamics and
kinematics, and \var{q} is an n-element vector of joint state.
For an n-axis manipulator \var{M} is an $n \times n$ symmetric 
matrix.

If \var{q} is a matrix each row is interpreted as a joint state
vector, and \var{I} is an $n \times n \times m$ matrix where \var{m} is the
number  of rows in \var{q}.

Note that if the \var{robot} contains motor inertia parameters then motor
inertia, referred to the link reference frame, will be added to the diagonal of
\var{M}.
}
\example{To show how the inertia `seen' by the waist joint varies as
a function of joint angles 2 and 3 the following code could be used.}
{\small
\begin{verbatim}
      >> [q2,q3] = meshgrid(-pi:0.2:pi, -pi:0.2:pi);
      >> q = [zeros(length(q2(:)),1) q2(:) q3(:) zeros(length(q2(:)),3)];
      >> I = inertia(p560, q);
      >> surfl(q2, q3, squeeze(I(1,1,:)));
\end{verbatim}}
\centerline{\psfig{figure=figs/surfl.eps,width=10cm}}

\msa{robot, rne, itorque, coriolis, gravload}
\mref{M.~W. Walker and D.~E. Orin.
\newblock Efficient dynamic computer simulation of robotic mechanisms.
\newblock {\em ASME Journal of Dynamic Systems, Measurement and Control},
  104:205--211, 1982.
  }

\vfil\eject 

%%%%%%%%%%%%%%%%%%%%%%%%%%%%%%%%%%%%%%%
\function{ishomog}
\mpur{Test if argument is a homogeneous transformation}

\msyn{
ishomog(x)
}

\mdes{Returns true if \var{x} is a 4 $\times$ 4 matrix.}
\vfil\eject 


%%%%%%%%%%%%%%%%%%%%%%%%%%%%%%%%%%%%%%%
\function{itorque}
\mpur{Compute the manipulator inertia torque component}

\msyn{
tau\_i = itorque(robot, q, qdd)
}

\mdes{\var{itorque} returns the joint torque due to inertia at the
specified pose \var{q} and acceleration \var{qdd} which is given by
\[
\vec{\tau}_i = {\bf M}(\vec{q}) \vec{\ddot{q}}
\]


If \var{q} and \var{qdd} are row vectors, \var{itorque} is a row vector 
of joint torques.
If \var{q} and \var{qdd} are matrices, each row is interpreted as a joint state 
vector, and \var{itorque} is a matrix in which each row is the inertia torque
for the corresponding rows of  \var{q} and \var{qdd}.

\var{robot} is a robot object that describes the kinematics and dynamics
of the manipulator and drive.
If  \var{robot} contains motor inertia parameters then motor
inertia, referred to the link reference frame, will be included
in the diagonal of \var{M} and influence the inertia torque result.
}


\msa{robot, rne, coriolis, inertia, gravload}
\vfil\eject 

%%%%%%%%%%%%%%%%%%%%%%%%%%%%%%%%%%%%%%%
\function{jacob0}
\mpur{Compute manipulator Jacobian in base coordinates}

\msyn{
jacob0(robot, q)
}

\mdes{\var{jacob0} returns a Jacobian matrix for the robot object
\var{robot} in the  pose \var{q} and as 
expressed in the base coordinate frame.

The manipulator Jacobian matrix, $ {}^0 {\bf J}_q$, maps differential
velocities in joint space, $\dot{q}$, to Cartesian velocity of the end-effector expressed
in the base coordinate frame.
\[
{}^0 \vec{\dot{x}} = {}^0 {\bf J}_q(\vec{q}) \vec{\dot{q}}
\]

For an n-axis manipulator the Jacobian is a $6 \times n$ matrix.}

\msa{jacobn, diff2tr, tr2diff, robot}
\mref{R.~P. Paul, B. Shimano and G.  E. Mayer.
\newblock {\em Kinematic Control Equations for Simple Manipulators}.
\newblock IEEE Systems, Man and Cybernetics 11(6), pp 449-455, June 1981.}
\vfil\eject 

%%%%%%%%%%%%%%%%%%%%%%%%%%%%%%%%%%%%%%%
\function{jacobn}
\mpur{Compute manipulator Jacobian in end-effector coordinates}

\msyn{
jacobn(robot, q)
}

\mdes{\var{jacobn} returns a Jacobian matrix for the robot object
\var{robot} in the  pose \var{q} and as 
expressed in the  end-effector coordinate frame.

The manipulator Jacobian matrix, $ {}^0 {\bf J}_q$, maps differential
velocities in joint space, $\dot{q}$, to Cartesian velocity of the end-effector expressed
in the end-effector
coordinate frame.
\[
{}^n \vec{\dot{x}} = {}^n {\bf J}_q(\vec{q}) \vec{\dot{q}}
\]
The relationship between tool-tip forces and joint torques is given
by
\[
\vec{\tau} = {}^n {\bf J}_q(\vec{q})^\prime {}^n \vec{F}
\]
For an n-axis manipulator the Jacobian is a $6 \times n$ matrix.}

\msa{jacob0, diff2tr, tr2diff, robot}
\mref{R.~P. Paul, B. Shimano and G.  E. Mayer.
\newblock {\em Kinematic Control Equations for Simple Manipulators}.
\newblock IEEE Systems, Man and Cybernetics 11(6), pp 449-455, June 1981.}
\vfil\eject 

%%%%%%%%%%%%%%%%%%%%%%%%%%%%%%%%%%%%%%%
\function{jtraj}
\mpur{Compute a joint space trajectory between two joint coordinate poses}

\msyn{
[q qd qdd] = jtraj(q0, q1, n)\\[0pt]
[q qd qdd] = jtraj(q0, q1, n, qd0, qd1)\\[0pt]
[q qd qdd] = jtraj(q0, q1, t)\\[0pt]
[q qd qdd] = jtraj(q0, q1, t, qd0, qd1)
}

\mdes{\var{jtraj} returns a joint space trajectory \var{q} from joint
coordinates  \var{q0} to \var{q1}.  The number
of points is \var{n} or the length of the given time vector \var{t}.  A 7th
order polynomial is used with default zero boundary conditions for
velocity and acceleration.

  Non-zero boundary velocities can be
optionally specified as \var{qd0} and \var{qd1}.

The trajectory is a matrix, with one row per time step, and
one column per joint.
The function can optionally return a velocity and acceleration
trajectories as \var{qd} and \var{qdd} respectively.
}
\msa{ctraj}

\vfil\eject 


%%%%%%%%%%%%%%%%%%%%%%%%%%%%%%%%%%%%%%%
\function{link}
\mpur{Link object}

\msyn{
L = link\\[0pt]
L = link([alpha, a, theta, d], convention)\\[0pt]
L = link([alpha, a, theta, d, sigma], convention)\\[0pt]
L = link(dyn\_row, convention)\\[0pt]
A = link(q)\\[0pt]
show(L)
}

\mdes{The \var{link} function constructs a \textbf{link} object.
The object contains kinematic and dynamic parameters as well as actuator and
transmission
parameters.
The first form returns a default object, while the second and third forms
initialize the kinematic model based on Denavit and Hartenberg parameters.
The dynamic model can be initialized using the fourth form of the constructor
where \var{dyn\_row} is a $1 \times 20$ matrix which is one row of the legacy
\var{dyn} matrix.

By default the standard Denavit and Hartenberg conventions are assumed but
this can be overridden by the optional \var{convention} argument which can
be set to either \var{'modified'} or \var{'standard'} (default).  Note
that any abbreviation of the string can be used, ie. \var{'mod'} or even
\var{'m'}.

The second last form given above is not a constructor but a link method that returns
the link transformation matrix for the given joint coordinate.
The argument is given to
the link object using parenthesis.  The single argument is taken as the link
variable $q$ and substituted for $\theta$ or $D$ for a revolute or prismatic
link respectively.

The  Denavit and Hartenberg parameters describe the spatial relationship between
this link and the previous one.  The meaning of the fields for each kinematic
convention are
summarized in the following table.

\begin{tabular}{lccl}
variable & DH & MDH & description \\ \hline
\var{alpha} & $\alpha_i$    & $\alpha_{i-1}$ & link twist angle \\
\var{A}     &  $A_i$        & $A_{i-1}$      & link length \\
\var{theta} &  $\theta_i$   & $\theta_i$     & link rotation angle \\
\var{D}     & $D_i$         & $D_i$          & link offset distance \\
\var{sigma} &  $\sigma_i$   & $\sigma_i$     & joint type; 0 for revolute, non-zero for prismatic \\
\end{tabular}


Since Matlab does not support the concept of public class variables methods have
been written to allow link object parameters to be referenced (r) or assigned
(a) as given by the following table
 
\begin{center}
\begin{tabular}{lcp{6cm}}
Method          & Operations & \multicolumn{1}{c|}{Returns} \\
\tt \textbf{link}.alpha	&r+a&  link twist angle \\	
\tt \textbf{link}.A		&r+a&   link length \\
\tt \textbf{link}.theta	&r+a&   link rotation angle \\
\tt \textbf{link}.D		&r+a&   link offset distance \\
\tt \textbf{link}.sigma	&r+a&  joint type; 0 for revolute, non-zero for prismatic\\
\tt \textbf{link}.RP		&r&  joint type; 'R' or 'P'\\
\tt \textbf{link}.mdh	&r+a& DH convention: 0 if standard, 1 if modified\\ \hline
\tt \textbf{link}.I		&r& $3 \times 3$ symmetric inertia matrix\\
\tt \textbf{link}.I		&a& assigned from a $3 \times 3$  matrix or a 6-element vector
interpretted as $[I_{xx} \, I_{yy} \, I_{zz} \, I_{xy} \, I_{yz} \, I_{xz} ]$\\
\tt \textbf{link}.m		&r+a& link mass\\
\tt \textbf{link}.r		&r+a& $3 \times 1$ link COG vector \\ \hline
\tt \textbf{link}.G		&r+a& gear ratio\\
\tt \textbf{link}.Jm		&r+a& motor inertia\\
\tt \textbf{link}.B		&r+a& viscous friction\\
\tt \textbf{link}.Tc		&r& Coulomb friction, $1\times 2$ vector
where $[\tau^+ \; \tau^-]$\\ 
\tt \textbf{link}.Tc		&a& Coulomb friction; for symmetric friction
this is a scalar, for asymmetric friction it is a 2-element vector for
positive and negative velocity \\ \hline
\tt \textbf{link}.dh		&r+a&  row of legacy DH matrix \\
\tt \textbf{link}.dyn	&r+a& row of legacy DYN matrix \\
\tt \textbf{link}.qlim	&r+a& joint coordinate limits, 2-vector \\
\tt \textbf{link}.islimit(q)	&r& return true if value of \var{q} is outside
			the joint limit bounds\\
\tt \textbf{link}.offset	&r+a& joint coordinate offset (see discussion for
\var{robot} object).
\end{tabular}
\end{center} 

The default is for standard Denavit-Hartenberg conventions, zero friction,
mass and inertias.

The \var{display} method gives a one-line summary of the link's kinematic
parameters.
The \var{show} method displays as many link parameters as have been initialized
for that link.
}

\mex{}
\begin{verbatim}
>> L = link([-pi/2, 0.02, 0, 0.15])
L =  
  -1.570796     0.020000        0.000000        0.150000        R      (std)
>> L.RP
ans =
  R
>> L.mdh
ans =
     0
>> L.G = 100;
>> L.Tc = 5;
>> L 
L =  
  -1.570796     0.020000        0.000000        0.150000        R     (std)
>> show(L)
alpha  = -1.5708
A      = 0.02
theta  = 0
D      = 0.15
sigma  = 0
mdh    = 0
G      = 100
Tc     = 5 -5
>> 
\end{verbatim}


\malg{For the standard Denavit-Hartenberg conventions the \hom
\[
{}^{i-1} {\bf A}_i = \left [ \begin{array}{cccc}
        \cos \theta_i & -\sin \theta_i \cos \alpha_i & \sin \theta_i \sin
        \alpha_i & a_i \cos \theta_i\\
        \sin \theta_i & \cos \theta_i \cos \alpha_i & -\cos \theta_i \sin
        \alpha_i & a_i \sin \theta_i\\
        0 & \sin \alpha_i & \cos \alpha_i & d_i\\
0 & 0 & 0 & 1
        \end{array} \right ]    \label{eq:Amatrix}
\]
represents each link's
coordinate frame with respect to the previous link's coordinate system.
For a revolute joint $\theta_i$ is offset by 

For the modified Denavit-Hartenberg conventions it is instead
\[
{}^{i-1} {\bf A}_i = \left [ \begin{array}{cccc}
        \cos \theta_i & -\sin \theta_i & 0 & a_{i-1}\\
        \sin \theta_i \cos \alpha_{i-1} & \cos \theta_i \cos \alpha_{i-1} & - \sin
        \alpha_{i-1} & -d_i \sin \alpha_{i-1}\\
        \sin \theta_i \sin \alpha_{i-1}  & \cos \theta_i \sin \alpha_{i-1} & \cos \alpha_{i-1} & d_i \cos \alpha_{i-1} \\
0 & 0 & 0 & 1
        \end{array} \right ]    \label{eq:Amatrix-m}
\]
}

\msa{showlink, robot}
\mref{R.~P. Paul.
\newblock {\em Robot Manipulators: Mathematics, Programming, and Control}.
\newblock MIT Press, Cambridge, Massachusetts, 1981.\\
J.~J. Craig, {\em Introduction to Robotics}.
\newblock Addison Wesley, second~ed., 1989.}
\vfil\eject 

%%%%%%%%%%%%%%%%%%%%%%%%%%%%%%%%%%%%%%%
\function{maniplty}
\mpur{Manipulability measure}

\msyn{
m = maniplty(robot, q)\\[0pt]
m = maniplty(robot, q, which)
}

\mdes{\var{maniplty} computes the scalar manipulability index for the manipulator
at the given pose.  Manipulability varies from 0 (bad) to 1 (good).
\var{robot} is a robot object that contains kinematic and optionally dynamic
parameters for the manipulator.
Two measures are supported and are selected by the optional argument
\var{which} can be either \var{'yoshikawa'} (default) or \var{'asada'}.
Yoshikawa's manipulability measure is based purely on kinematic data, and gives
an indication of how `far' the manipulator is from singularities and thus able
to move and exert forces uniformly in all directions. 

Asada's manipulability measure utilizes manipulator dynamic data, and indicates 
how close the inertia ellipsoid is to spherical.

If \var{q} is a vector \var{maniplty} returns a scalar manipulability index.
If \var{q} is a matrix \var{maniplty }returns a column vector and each
row is the manipulability index for the pose specified by the corresponding
row of \var{q}.
}


\malg{Yoshikawa's measure is based on the condition number of the manipulator
Jacobian
\[
\eta_{yoshi} = \sqrt{| {\bf J}(\vec{q}) {\bf J}(\vec{q})'|}
\]
Asada's measure is computed from the Cartesian inertia matrix
\[
{\bf M}(\vec{x}) = {\bf J}(\vec{q})^{-T} {\bf M}(\vec{q}) {\bf J}(\vec{q})^{-1}
\]
The Cartesian manipulator inertia ellipsoid is
\[
\vec{x}' {\bf M}(\vec{x}) \vec{x} = 1
\]
and gives an indication of how well the manipulator can accelerate
in each of the Cartesian directions.  The scalar measure computed
here is the ratio of the smallest/largest ellipsoid axes
\[
\eta_{asada} = \frac{\min x}{\max x}
\]
Ideally the ellipsoid would be spherical, giving a ratio of 1, but in
practice will be less than 1.}

\msa{jacob0, inertia,robot}
\mref{T.~Yoshikawa, ``Analysis and control of robot manipulators with redundancy,''
  in {\em Proc. 1st Int. Symp. Robotics Research}, (Bretton Woods, NH),
    pp.~735--747, 1983.}

\vfil\eject 

%%%%%%%%%%%%%%%%%%%%%%%%%%%%%%%%%%%%%%%
\function{robot/nofriction}
\mpur{Remove friction from robot object}

\msyn{
robot2 = nofriction(robot)\\[0pt]
robot2 = nofriction(robot, 'all')\\[0pt]
}

\mdes{Return a new robot object with modified joint friction properties.
The first form sets the Coulomb friction values to zero in the
constituent links
The second form sets viscous and Coulomb friction values in the constituent 
links are set to zero.

The resulting robot object has its name string prepended with 'NF/'.

This is important for forward dynamics computation (\var{fdyn()})
where the presence of friction can prevent the numerical integration
from converging.
}

\msa{link/nofriction, robot, link, friction, fdyn}

\vfil\eject 

%%%%%%%%%%%%%%%%%%%%%%%%%%%%%%%%%%%%%%%
\function{link/nofriction}
\mpur{Remove friction from link object}

\msyn{
link2 = nofriction(link)\\[0pt]
link2 = nofriction(link, 'all')\\[0pt]
}

\mdes{Return a new link object with modified joint friction properties.
The first form sets the Coulomb friction values to zero.
The second form sets both viscous and Coulomb friction values to zero.

This is important for forward dynamics computation (\var{fdyn()})
where the presence of friction can prevent the numerical integration
from converging.
}

\msa{robot/nofriction, link, friction, fdyn}

\vfil\eject 

%%%%%%%%%%%%%%%%%%%%%%%%%%%%%%%%%%%%%%%
\function{oa2tr}
\mpur{Convert OA vectors to \hom}

\msyn{
oa2tr(o, a)
}

\mdes{\var{oa2tr} returns a rotational homogeneous transformation specified in
terms of the Cartesian orientation and approach vectors \var{o} and
\var{a} respectively.}

\malg{\[
{\bf T} = \left[ \begin{array}{cccc}
	\vec{\hat{o}} \times \vec{\hat{a}} & \vec{\hat{o}} & \vec{\hat{a}} & 0 \\
	0 & 0 & 0 & 1 \end{array} \right]
\]
where $\vec{\hat{o}}$ and $\vec{\hat{a}}$ are unit vectors corresponding
to \var{o} and \var{a} respectively.}

\msa{rpy2tr, eul2tr}
\vfil\eject 

%%%%%%%%%%%%%%%%%%%%%%%%%%%%%%%%%%%%%%%
\function{perturb}
\mpur{Perturb robot dynamic parameters}

\msyn{
robot2 = perturb(robot, p)
}

\mdes{Return a new robot object with randomly modified dynamic parameters:
link mass and inertia.
The perturbation is multiplicative so that
values are multiplied by random numbers in the interval (1-p) to (1+p).

Useful for investigating the robustness of various model-based control 
schemes where one model forms the basis of the model-based controller and
the peturbed model is used for the actual plant.

The resulting robot object has its name string prepended with 'P/'.
}

\msa{fdyn, rne, robot}

\vfil\eject 



%%%%%%%%%%%%%%%%%%%%%%%%%%%%%%%%%%%%%%%
\function{puma560}
\mpur{Create a Puma 560 robot object}

\msyn{
puma560
}

\mdes{Creates the \textbf{robot} object \var{p560} which describes the kinematic and dynamic
characteristics of a Unimation Puma 560 manipulator.
The kinematic conventions used are as per Paul and Zhang, and
all quantities are in standard SI units.

Also defines the joint coordinate vectors \var{qz}, \var{qr} and \var{qstretch} 
corresponding to the zero-angle, ready and fully extended (in X-direction)
poses respectively.}

\centerline{\psfig{figure=figs/pumaaxis.eps,height=8cm}}
Details of coordinate frames used for the Puma 560 shown here in its zero angle pose.

\msa{robot, puma560akb, stanford}
\mref{R.~P. Paul and H.~Zhang, ``Computationally efficient kinematics for
  manipulators with spherical wrists,'' {\em Int. J. Robot. Res.}, vol.~5,
    no.~2, pp.~32--44, 1986.
    
P.~Corke and B.~Armstrong-H\'{e}louvry, ``A search for consensus among model
  parameters reported for the {PUMA} 560 robot,'' in {\em Proc. IEEE Int. Conf.
    Robotics and Automation}, (San Diego), pp.~1608--1613, May 1994.
    
P.~Corke and B.~Armstrong-H\'{e}louvry, ``A meta-study of {PUMA} 560 dynamics:
  A critical appraisal of literature data,'' {\em Robotica}, vol.~13, no.~3,
    pp.~253--258, 1995.
    }

\vfil\eject 

%%%%%%%%%%%%%%%%%%%%%%%%%%%%%%%%%%%%%%%
\function{puma560akb}
\mpur{Create a Puma 560 robot object}

\msyn{
puma560akb
}

\mdes{Creates the \textbf{robot} object \var{p560m} which describes the 
kinematic and dynamic
characteristics of a Unimation Puma 560 manipulator.
It uses Craig's modified Denavit-Hartenberg notation with the
particular kinematic conventions from Armstrong, Khatib and Burdick.
All quantities are in standard SI units.

Also defines the joint coordinate vectors \var{qz}, \var{qr} and \var{qstretch} 
corresponding to the zero-angle, ready and fully extended 
(in X-direction) poses respectively.}

\msa{robot, puma560, stanford}
\mref{
B.~Armstrong, O.~Khatib, and J.~Burdick, ``The explicit dynamic model and
  inertial parameters of the {Puma} 560 arm,'' in {\em Proc. IEEE Int. Conf.
    Robotics and Automation}, vol.~1, (Washington, USA), pp.~510--18, 1986.

}
\vfil\eject 

%%%%%%%%%%%%%%%%%%%%%%%%%%%%%%%%%%%%%%%
\function{qinterp}
\mpur{Interpolate unit-quaternions}

\msyn{
QI = qinterp(Q1, Q2, r)
}

\mdes{Return a unit-quaternion that interpolates between Q1 and Q2 as 
\var{r} varies between 0 and 1 inclusively.
This is a spherical linear interpolation (slerp) that can
be interpreted as interpolation along a great circle arc on a sphere.

If \var{r} is a vector, then a cell array of quaternions is returned 
corresponding to successive values of \var{r}.}

\mex{A simple example}
\begin{verbatim}
>> q1 = quaternion(rotx(0.3))
 
q1 = 
  0.98877 <0.14944, 0, 0>
 
>> q2 = quaternion(roty(-0.5))
 
q2 = 
  0.96891 <0, -0.2474, 0>
 
>> qinterp(q1, q2, 0)
 
ans = 
  0.98877 <0.14944, 0, 0>
 
>> qinterp(q1, q2, 1)
 
ans = 
  0.96891 <0, -0.2474, 0>
 
>> qinterp(q1, q2, 0.3)
 
ans = 
  0.99159 <0.10536, -0.075182, 0>
 
>> 
\end{verbatim}

\mref{K.~Shoemake, ``Animating rotation with quaternion curves.,'' in {\em
  Proceedings of ACM SIGGRAPH}, (San Francisco), pp.~245--254, The Singer
    Company, Link Flight Simulator Division, 1985.}
\vfil\eject 


%%%%%%%%%%%%%%%%%%%%%%%%%%%%%%%%%%%%%%%
\function{quaternion}
\mpur{Quaternion object}

\msyn{
q = quaternion(qq)\\[0pt]
q = quaternion(v, theta)\\[0pt]
q = quaternion(R)\\[0pt]
q = quaternion([s vx vy vz])
}

\mdes{\var{quaternion} is the constructor for a \textbf{quaternion} object.
The first form returns a new
object with the same value as its argument.
The second form initializes the quaternion to a rotation of \var{theta}
about the vector \var{v}.  }

\mex{A simple example}
\begin{verbatim}
>> quaternion(1, [1 0 0])
 
ans = 
  0.87758 <0.47943, 0, 0>
 
>> quaternion( rotx(1) )
 
ans = 
  0.87758 <0.47943, 0, 0>
 
>> 
\end{verbatim}

The third form sets the quaternion to a rotation equivalent to the
given $3\times3$ rotation matrix, or the rotation submatrix of a
$4\times 4$ \hom.

The fourth form sets the four quaternion elements directly where \var{s}
is the scalar component and \var{[vx vy vz]} the vector.

All forms, except the last, return a unit quaternion, ie. one whose
magnitude is unity.

Some operators are overloaded for the quaternion class

\begin{tabular}{lp{8cm}}
\var{q1 * q2} & returns quaternion product or compounding \\
\var{q * v}   & returns a quaternion vector product, that is the vector $v$ is 
		rotated by the quaternion.  $v$ is a $ 3 \times 3$ vector \\
\var{q1 / q2} &  returns $q_1 * q_2^{-1}$ \\
\var{q}$\wedge$\var{j}    & returns $q^j$ where $j$ is an integer exponent.  For $j>0$ the 
		result is obtained by repeated multiplication.  
		For $j<0$ the final result is inverted.\\
\var{double(q)} & returns the quaternion coeffients as a 4-element row vector \\
\var{inv(q)} & returns the quaterion inverse \\
\var{norm(q)} & returns the quaterion magnitude \\
\var{plot(q)} & displays a 3D plot showing the standard coordinate frame after
rotation by \var{q}.\\
\var{unit(q)} & returns the corresponding unit quaterion 
\end{tabular}

Some public class variables methods are also available for reference only.
 
\begin{center}
\begin{tabular}{lp{6cm}}
method        & \multicolumn{1}{c}{Returns} \\ \hline
\tt \textbf{quaternion}.d    &  return 4-vector of quaternion elements \\        
\tt \textbf{quaternion}.s    &  return scalar component \\        
\tt \textbf{quaternion}.v    &  return vector component \\        
\tt \textbf{quaternion}.t    &  return equivalent homogeneous transformation matrix \\        
\tt \textbf{quaternion}.r    &  return equivalent orthonormal rotation matrix 
\end{tabular}
\end{center} 
}

\mex{}
\begin{verbatim}
>> t = rotx(0.2) 
t =
    1.0000         0         0         0
         0    0.9801   -0.1987         0
         0    0.1987    0.9801         0
         0         0         0    1.0000
>> q1 = quaternion(t) 
q1 = 
  0.995 <0.099833, 0, 0>

>> q1.r
ans =
    1.0000         0         0
         0    0.9801   -0.1987
         0    0.1987    0.9801

>> q2 = quaternion( roty(0.3) ) 
q2 = 
  0.98877 <0, 0.14944, 0>

>> q1 * q2
ans = 
  0.98383 <0.098712, 0.14869, 0.014919>

>> q1*q1 
ans = 
  0.98007 <0.19867, 0, 0>

>> q1^2 
ans = 
  0.98007 <0.19867, 0, 0>

>> q1*inv(q1)
ans = 
  1 <0, 0, 0>
 
>> q1/q1
ans = 
  1 <0, 0, 0>
 
>> q1/q2
ans = 
  0.98383 <0.098712, -0.14869, -0.014919>
  
>> q1*q2^-1
ans = 
  0.98383 <0.098712, -0.14869, -0.014919>

\end{verbatim}

\mcau{At the moment vectors or arrays of quaternions are not supported.
You can however use cell arrays to hold a number of quaternions.}

\msa{quaternion/plot}	
\mref{K.~Shoemake, ``Animating rotation with quaternion curves.,'' in {\em
  Proceedings of ACM SIGGRAPH}, (San Francisco), pp.~245--254, The Singer
    Company, Link Flight Simulator Division, 1985.}
\vfil\eject 


%%%%%%%%%%%%%%%%%%%%%%%%%%%%%%%%%%%%%%%
\function{quaternion/plot}
\mpur{Plot quaternion rotation}

\msyn{
plot(Q)
}

\mdes{\var{plot} is overloaded for \textbf{quaternion} objects
and displays a 3D plot which shows how the standard axes are transformed
under that rotation.}

\mex{A rotation of 0.3rad about the X axis.  Clearly the X axis is 
invariant under this rotation.}
\begin{verbatim}
>> q=quaternion(rotx(0.3))
 
q = 
 
  0.85303<0.52185, 0, 0>
 
>> plot(q)
\end{verbatim}

\centerline{\psfig{figure=figs/quat.eps,width=10cm}}


\msa{quaternion}
\vfil\eject 

%%%%%%%%%%%%%%%%%%%%%%%%%%%%%%%%%%%%%%%
\function{rne}
\mpur{Compute inverse dynamics via recursive Newton-Euler formulation}

\msyn{
tau = rne(robot, q, qd, qdd)\\[0pt]
tau = rne(robot, [q qd qdd])\\[0pt]

tau = rne(robot, q, qd, qdd, grav)\\[0pt]
tau = rne(robot, [q qd qdd], grav)\\[0pt]

tau = rne(robot, q, qd, qdd, grav, fext)\\[0pt]
tau = rne(robot, [q qd qdd], grav, fext)
}

\mdes{\var{rne} computes the equations of motion in an efficient manner, giving
joint torque as a function of 
joint position, velocity and acceleration.

If \var{q}, \var{qd} and \var{qdd} are row vectors then \var{tau} is a row
vector of joint torques.
If  \var{q}, \var{qd} and \var{qdd} are matrices then \var{tau} is a matrix
in which each row is the joint torque for the corresponding rows of
 \var{q}, \var{qd} and \var{qdd}.

Gravity direction is defined by the robot object
but may be overridden by providing a gravity acceleration
vector \var{grav = [gx gy gz]}.

An external force/moment acting on the end of the manipulator may also be
specified by a 6-element vector \var{fext = [Fx Fy Fz Mx My Mz]} in the
end-effector coordinate frame.

The torque computed may contain contributions due to armature
inertia and joint friction if these are specified in the parameter matrix \var{dyn}.

The MEX-file version of this function is over 1000 times faster than the
M-file.  See Section 1 of this manual for information about how to compile
and install the MEX-file.}

\malg{
Coumputes the joint torque
\[
\vec{\tau} =
{\bf M}(\vec{q})\ddot{\vec{q}} + {\bf
C}(\vec{q},\vec{\dot{q}}) \vec{\dot{q}} +
{\bf F}(\vec{\dot{q}}) + {\bf G}(\vec{q})
\]
where $\bf M$ is the manipulator inertia matrix, $\bf C$ is the Coriolis
and centripetal torque, $\bf F$ the viscous and Coulomb friction, and $\bf G$
the gravity load.}

\mcau{
The MEX file currently ignores support base and tool transforms.}

\msa{robot, fdyn, accel, gravload, inertia, friction}

\mlim{A MEX file is currently only available for Sparc architecture.}
\mref{J.~Y.~S. Luh, M.~W. Walker, and R.~P.~C. Paul.
\newblock On-line computational scheme for mechanical manipulators.
\newblock {\em ASME Journal of Dynamic Systems, Measurement and Control},
  102:69--76, 1980.
}

\vfil\eject 

%%%%%%%%%%%%%%%%%%%%%%%%%%%%%%%%%%%%%%%
\function{robot}
\mpur{Robot object}

\msyn{
r = robot\\[0pt]
r = robot(rr)\\[0pt]
r = robot(link ...)\\[0pt]
r = robot(DH ...)\\[0pt]
r = robot(DYN ...)
}

\mdes{\var{robot} is the constructor for a robot object.
The first form creates a default robot, and the second form returns a new
robot object with the same value as its argument.
The third form creates a robot from a cell array of link objects which define
the robot's kinematics and optionally dynamics.
The fourth and fifth forms create a robot object from legacy DH and DYN format
matrices.

The last three forms all accept optional trailing string arguments which are
taken in order as being robot name, manufacturer and comment.

Since Matlab does not support the concept of public class variables methods have
been written to allow robot object parameters to be referenced (r) or assigned
(a) as given by the following table
 
\begin{center}
\begin{tabular}{lcp{6cm}}
method        & Operation & \multicolumn{1}{c}{Returns} \\
\tt \textbf{robot}.n       &r   &  number of joints \\        
\tt \textbf{robot}.link    &r+a &   cell array of link objects \\
\tt \textbf{robot}.name    &r+a &   robot name string \\
\tt \textbf{robot}.manuf   &r+a &   robot manufacturer string \\
\tt \textbf{robot}.comment &r+a &  general comment string\\
\tt \textbf{robot}.gravity &r+a   &  3-element vector defining gravity direction\\
\tt \textbf{robot}.mdh     &r   & DH convention: 0 if standard, 1 if modified.  Determined
from the link objects.\\ \hline
\tt \textbf{robot}.base    &r+a & \hom\ defining base of robot\\
\tt \textbf{robot}.tool    &r+a & \hom\ defining tool of robot\\
\tt \textbf{robot}.dh      &r   &  legacy DH matrix \\
\tt \textbf{robot}.dyn     &r   & legacy DYN matrix \\
\tt \textbf{robot}.q  &r+a   & joint coordinates \\
\tt \textbf{robot}.qlim  &r+a   & joint coordinate limits, $n \times 2$ matrix \\
\tt \textbf{robot}.islimit  &r   & joint limit vector, for each joint set to
-1, 0 or 1 depending if below low limit, OK, or greater than upper limit\\
\tt \textbf{robot}.offset  &r+a   & joint coordinate offsets \\
\tt \textbf{robot}.plotopt  &r+a   & options for \var{plot()}\\
\tt \textbf{robot}.lineopt  &r+a   & line style for robot graphical links \\
\tt \textbf{robot}.shadowopt  &r+a   & line style for robot shadow links \\
\tt \textbf{robot}.handle  &r+a   & graphics handles
\end{tabular}
\end{center} 

Some of these operations at the robot level are actually wrappers around
similarly named link object functions: \var{offset},
\var{qlim}, \var{islimit}.

The offset vector is added to the user specified joint angles before
any kinematic or dynamic function is invoked (it is actually implemented
within the link object).  Similarly it is subtracted after an operation
such as inverse kinematics.  The need for a joint offset vector arises
because of the constraints of the Denavit-Hartenberg (or modified
Denavit-Hartenberg) notation.  The pose of the robot with zero joint
angles is frequently some rather unusual (or even unachievable) pose.
The joint coordinate offset provides a means to make an arbitrary pose
correspond to the zero joint angle case.

Default values for robot parameters are:

\begin{center}
\begin{tabular}{ll}
\textbf{robot}.name    & 'noname' \\
\textbf{robot}.manuf   & '' \\
\textbf{robot}.comment & '' \\
\textbf{robot}.gravity & $ [0 \,0 \,9.81]\, {\rm m/s^2}$ \\
\textbf{robot}.offset    & \var{ones(n,1)} \\
\textbf{robot}.base    & \var{eye(4,4)} \\
\textbf{robot}.tool    & \var{eye(4,4)} \\
\textbf{robot}.lineopt  & 'Color', 'black', 'Linewidth', 4 \\
\textbf{robot}.shadowopt  & 'Color', 'black', 'Linewidth', 1
\end{tabular}
\end{center}

The multiplication operator, \var{*}, is overloaded and the product of two robot
objects is a robot which is the series connection of the multiplicands.
Tool transforms of all but the last robot are ignored, base transform of
all but the first robot are ignored.

The \var{plot} function is also overloaded and is used to provide a robot
animation.  
}

\mex{}
\begin{verbatim}
>> L{1} = link([ pi/2  0  0  0])
L = 
    [1x1 link]

>> L{2} = link([ 0  0 0.5 0])
L = 
    [1x1 link]    [1x1 link]

>> r = robot(L)

r = 
 (2 axis, RR)
                grav = [0.00 0.00 9.81]
                standard D&H parameters

  alpha           A               theta           D             R/P
1.570796        0.000000        0.000000        0.000000        R     (std)
0.000000        0.000000        0.500000        0.000000        R     (std)

>> 
\end{verbatim}

\msa{link,plot}
\vfil\eject 


%%%%%%%%%%%%%%%%%%%%%%%%%%%%%%%%%%%%%%%
\function{robot/plot}
\mpur{Graphical robot animation}

\msyn{
plot(robot, q)\\[0pt]
plot(robot, q, arguments...)\\[0pt]
}

\centerline{\psfig{figure=figs/plot.eps,width=12cm}}

\mdes{\var{plot} is overloaded for \textbf{robot} objects
and displays a graphical representation of the robot given
the kinematic information in \var{robot}.
The robot is represented by a simple stick figure polyline where line 
segments join
the origins of the link coordinate frames.
If \var{q} is a matrix representing a joint-space trajectory then an animation
of the robot motion is shown.


{\tmsss GRAPHICAL ANNOTATIONS}

The basic stick figure robot can be annotated with
\begin{itemize}
\item shadow on the `floor'
\item XYZ wrist axes and labels,
shown by 3 short orthogonal line segments
which are colored: red (X or normal), green (Y or orientation) and blue (Z or
approach).  They can be optionally labelled XYZ or NOA.
\item joints,
these are 3D cylinders for 
revolute joints and boxes for prismatic joints\\
\item the robot's name
\end{itemize}

All of these require some kind of dimension and this is determined
using a simple heuristic from the workspace dimensions.  This dimension
can be changed by setting the multiplicative scale factor using the
\var{mag} option below.  These various annotations do slow the rate
at which animations will be rendered.

{\tmsss OPTIONS}

Options are specified by a variable length argument list comprising strings
and numeric values.  The allowed values are:}

\begin{tabular}{l p{4in}}
\vtt workspace \var{w} & set the 3D plot bounds or workspace to the matrix
\var{[xmin xmax ymin ymax zmin zmax]}\\
\vtt perspective & show a perspective view \\
\vtt ortho & show an orthogonal view \\
\vtt base, nobase & control display of base, a line from the floor upto joint 0\\
\vtt wrist, nowrist & control display of wrist axes\\
\vtt name, noname & control display of robot name near joint 0\\
\vtt shadow, noshadow & control display of a 'shadow' on the floor\\
\vtt joints, nojoints & control display of joints, these are cylinders for 
revolute joints and boxes for prismatic joints\\
\vtt xyz & wrist axis labels are X, Y, Z\\
\vtt noa & wrist axis labels are N, O, A\\
\vtt mag \var{scale} & annotation scale factor \\
\vtt erase, noerase & control erasure of robot after each change\\
\vtt loop, noloop & control whether animation is repeated endlessly
\end{tabular}

The options come from 3 sources and are processed in the order:
\begin{enumerate}
\item Cell array of options returned by the function \var{PLOTBOTOPT} if
found on the user's current path.
\item Cell array of options returned by the \var{.plotopt} method of the
\var{robot} object.  These are set by the \var{.plotopt} method.
\item  List of arguments in the command line.
\end{enumerate}

{\tmsss GETTING GRAPHICAL ROBOT STATE}

Each graphical robot has a unique tag set equal to the robot's name.
When \var{plot} is called it looks for all graphical objects with that name
and moves them.  The graphical robot holds a copy of the \var{robot} object
as \var{UserData}.  That copy contains the graphical handles of all the
graphical sub-elements of the robot and also the current joint angle state.

This state is used, and adjusted, by the \var{drivebot} function.
The current joint angle state can be obtained by
\var{q = plot(robot)}.
If multiple instances exist, that of the first one returned by \var{findobj()}
is given.

\mex{To animate two Pumas moving in the same figure window.}
\begin{verbatim}
>> clf
>> p560b = p560;                         % duplicate the robot
>> p560b.name = 'Another Puma 560';      % give it a unique name
>> p560b.base = transl([-.05 0.5 0]);    % move its base
>> plot(p560, qr);   % display it at ready position 
>> hold on
>> plot(p560b, qr);  % display it at ready position
>> t = [0:0.056:10];
>> jt = jtraj(qr, qstretch, t);  % trajectory to stretch pose
>> for q = jt', % for all points on the path
>>    plot(p560, q); 
>>    plot(p560b, q);
>> end
\end{verbatim}

To show multiple views of the same robot.
\begin{verbatim}
>> clf
>> figure             % create a new figure
>> plot(p560, qz);    % add a graphical robot
>> figure             % create another figure
>> plot(p560, qz);    % add a graphical robot
>> plot(p560, qr);    % both robots should move
\end{verbatim}

Now the two figures can be adjusted to give different viewpoints, for
instance, plan and elevation.

\mcau{\var{plot()} options are only processed on the first call when the
graphical object is established, they are skipped on subsequent calls.
Thus if you wish to change options, clear the figure before replotting.}

\msa{drivebot, fkine, robot}
\vfil\eject 
%%%%%%%%%%%%%%%%%%%%%%%%%%%%%%%%%%%%%%%
\function{rotvec}
\mpur{Rotation about a vector}

\msyn{
T = rotvec(v, theta)
}

\mdes{\var{rotvec} returns a homogeneous transformation representing a 
rotation of \var{theta} radians about the vector \var{v}.}

\msa{rotx, roty, rotz}
\vfil\eject 

%%%%%%%%%%%%%%%%%%%%%%%%%%%%%%%%%%%%%%%
\function{rotx,roty,rotz}
\mpur{Rotation about X, Y or Z axis}

\msyn{
T = rotx(theta)\\[0pt]
T = roty(theta)\\[0pt]
T = rotz(theta)
}

\mdes{Return a homogeneous transformation representing a 
rotation of \var{theta} radians about the X, Y or Z axes.}

\msa{rotvec}
\vfil\eject 

%%%%%%%%%%%%%%%%%%%%%%%%%%%%%%%%%%%%%%%
\function{rpy2tr}
\mpur{Roll/pitch/yaw angles to \hom}

\msyn{
T = rpy2tr([r p y])\\[0pt]
T = rpy2tr(r,p,y)
}

\mdes{\var{rpy2tr}
returns a homogeneous transformation for the specified
roll/pitch/yaw angles in radians.
\[
T = R_Z(r) R_Y(p) R_X(y)
\]
}

\msa{tr2rpy, eul2tr}
\mref{R.~P. Paul, {\em Robot Manipulators: Mathematics, Programming, and Control}.
\newblock Cambridge, Massachusetts: MIT Press, 1981.}
\vfil\eject 

%%%%%%%%%%%%%%%%%%%%%%%%%%%%%%%%%%%%%%%
\function{rtdemo}
\mpur{Robot Toolbox demonstration}

\msyn{
rtdemo
}

\mdes{This script provides demonstrations for most functions within
the Robotics Toolbox.
}

\mcau{This script clears all variables in the workspace and deletes
all figures.}
\vfil\eject 

%%%%%%%%%%%%%%%%%%%%%%%%%%%%%%%%%%%%%%%
\function{showlink}
\mpur{Show robot link details}

\msyn{
showlink(robot)\\[0pt]
showlink(link)
}

\mdes{
Displays in detail all the parameters, including all defined
inertial parameters, of a link.  The first form provides this
level of detail for all links in the specified manipulator.
roll/pitch/yaw angles in radians.
}
\mex{To show details of Puma link 2}
\begin{verbatim}
>> showlink(p560.link{2})
alpha  = 0
A      = 0.4318
theta  = 0
D      = 0
sigma  = 0
mdh    = 0
offset = 0
m      = 17.4
r      = -0.3638
           0.006
          0.2275
I      = 0.13           0           0
            0       0.524           0
            0           0       0.539
Jm     = 0.0002
G      = 107.815
B      = 0.000817
Tc     = 0.126      -0.071
qlim   = 
>> 
\end{verbatim}

\vfil\eject 


%%%%%%%%%%%%%%%%%%%%%%%%%%%%%%%%%%%%%%%
\function{stanford}
\mpur{Create a Stanford manipulator robot object }

\msyn{
stanford
}

\centerline{\psfig{figure=figs/stanford.eps,width=12cm}}

\mdes{Creates the \textbf{robot} object \var{stan} which describes the kinematic and dynamic
characteristics of a Stanford manipulator.
Specifies armature inertia and gear ratios.
All quantities are in standard SI units.}


\msa{robot, puma560}
\mref{R.~Paul, ``Modeling, trajectory calculation and servoing of a computer
  controlled arm,'' Tech. Rep. {AIM}-177, Stanford University, Artificial
    Intelligence Laboratory, 1972.

R.~P. Paul, {\em Robot Manipulators: Mathematics, Programming, and Control}.
\newblock Cambridge, Massachusetts: MIT Press, 1981.}
\vfil\eject 


%%%%%%%%%%%%%%%%%%%%%%%%%%%%%%%%%%%%%%%
\function{tr2diff}
\mpur{Convert a \hom\ to a differential motion vector}

\msyn{
d = tr2diff(T)\\[0pt]
d = tr2diff(T1, T2)\\[0pt]
}

\mdes{The first form of \var{tr2diff} returns a 6-element differential 
motion vector representing the incremental translation and rotation described by
the homogeneous transform \var{T}.  It is assumed that \var{T} is of the
form
\[
\left[ \begin{array}{cccc}
	0 &	 -\delta_z &	 \delta_y &	d_x \\
	\delta_z &	0 &	-\delta_x &	d_y \\
	-\delta_y &	\delta_x & 0 &		d_z \\
	0 &		0 &	0 &		0
	\end{array}
\right]
\]
The translational elements of \var{d} are assigned directly.  The rotational
elements are computed from the mean of the two values that appear in the
skew-symmetric matrix.

The second form of \var{tr2diff} returns a 6-element differential
motion vector representing the displacement from \var{T1} to \var{T2},
that is,  \var{T2} - \var{T1}.
\[
d = \left[ \begin{array}{c}
	\vec{p}_2 - \vec{p}_1 \\
	1/2 \left( \vec{n}_1 \times \vec{n}_2 + \vec{o}_1 \times \vec{o}_2 +
		\vec{a}_1 \times \vec{a}_2 \right) \end{array} \right]
\]
}

\msa{diff2tr}
\mref{R.~P. Paul, {\em Robot Manipulators: Mathematics, Programming, and Control}.
\newblock Cambridge, Massachusetts: MIT Press, 1981.}
\vfil\eject 


%%%%%%%%%%%%%%%%%%%%%%%%%%%%%%%%%%%%%%%
\function{tr2eul}
\mpur{Convert a homogeneous transform to Euler angles}

\msyn{
[a b c] = tr2eul(T)
}

\mdes{\var{tr2eul} returns a vector of Euler angles, in radians,
corresponding
to the rotational part of the homogeneous transform \var{T}.
\[
T_{rot} = R_Z(a) R_Y(b) R_Z(c)
\]
}
\mcau{Note that 12 different Euler angle sets or conventions exist.  The
convention used here is the common one for robotics, but is not the one used
for example in the aerospace community.}

\msa{eul2tr, tr2rpy}
\mref{R.~P. Paul, {\em Robot Manipulators: Mathematics, Programming, and Control}.
\newblock Cambridge, Massachusetts: MIT Press, 1981.}
\vfil\eject 

%%%%%%%%%%%%%%%%%%%%%%%%%%%%%%%%%%%%%%%
\function{tr2jac}
\mpur{Compute a Jacobian to map differential motion between frames}

\msyn{
jac = tr2jac(T)
}

\mdes{\var{tr2jac} returns a $6 \times 6$ Jacobian matrix to map 
differential motions or velocities between
frames related by the homogeneous transform \var{T}.

If \var{T} represents a homogeneous transformation from frame A to frame B,
${}^A {\bf T}_B$, then
\[
	{}^B \vec{\dot{x}} = {}^B {\bf J}_A {}^A \vec{\dot{x}}
\]
where ${}^B {\bf J}_A$ is given by \var{tr2jac(T)}.
}

\msa{tr2diff, diff2tr}
\mref{R.~P. Paul, {\em Robot Manipulators: Mathematics, Programming, and Control}.
\newblock Cambridge, Massachusetts: MIT Press, 1981.}
\vfil\eject 


%%%%%%%%%%%%%%%%%%%%%%%%%%%%%%%%%%%%%%%
\function{tr2rpy}
\mpur{Convert a homogeneous transform to roll/pitch/yaw angles}

\msyn{
[a b c] = tr2rpy(T)
}

\mdes{\var{tr2rpy} returns a vector of roll/pitch/yaw angles, in radians,
corresponding to the rotational part of the homogeneous transform \var{T}
\[
T_{rot} = R_Z(r) R_Y(p) R_X(y)
\]
}

\msa{rpy2tr, tr2eul}
\mref{R.~P. Paul, {\em Robot Manipulators: Mathematics, Programming, and Control}.
\newblock Cambridge, Massachusetts: MIT Press, 1981.}
\vfil\eject 

%%%%%%%%%%%%%%%%%%%%%%%%%%%%%%%%%%%%%%%
\function{transl}
\mpur{Translational transformation}

\msyn{
T = transl(x, y, z)\\[0pt]
T = transl(v)\\[0pt]
v = transl(T)\\[0pt]
xyz = transl(TC)
}

\mdes{The first two forms return a homogeneous transformation representing a 
translation expressed as three scalar \var{x}, \var{y} and \var{z},
or a Cartesian vector \var{v}.

The third form returns the translational part of a homogeneous 
transform as a 3-element column vector.

The fourth form returns a matrix whose columns are the X, Y and Z columns
of the $4 \times 4 \times m$ Cartesian trajectory matrix \var{TC}.}


\msa{ctraj}
\vfil\eject 

%%%%%%%%%%%%%%%%%%%%%%%%%%%%%%%%%%%%%%%
\function{trinterp}
\mpur{Interpolate \hom s}

\msyn{
T = trinterp(T0, T1, r)
}

\mdes{\var{trinterp} interpolates between the two \hom s
\var{T0} and \var{T1} as \var{r} varies between 0 and 1 inclusively.
This is generally used for computing straight line or `Cartesian'
motion.  Rotational interpolation is achieved using quaternion
spherical linear interpolation.
}

\mex{Interpolation of homogeneous transformations.}
\begin{verbatim}
>> t1=rotx(.2)              

t1 =
    1.0000         0         0         0
         0    0.9801   -0.1987         0
         0    0.1987    0.9801         0
         0         0         0    1.0000

>> t2=transl(1,4,5)*roty(0.3)         

t2 =
    0.9553         0    0.2955    1.0000
         0    1.0000         0    4.0000
   -0.2955         0    0.9553    5.0000
         0         0         0    1.0000

>> trinterp(t1,t2,0)	% should be t1

ans =
    1.0000         0         0         0
         0    0.9801   -0.1987         0
         0    0.1987    0.9801         0
         0         0         0    1.0000

>> trinterp(t1,t2,1)   % should be t2

ans =
    0.9553         0    0.2955    1.0000
         0    1.0000         0    4.0000
   -0.2955         0    0.9553    5.0000
         0         0         0    1.0000

>> trinterp(t1,t2,0.5)	% 'half way' in between

ans =
    0.9887    0.0075    0.1494    0.5000
    0.0075    0.9950   -0.0998    2.0000
   -0.1494    0.0998    0.9837    2.5000
         0         0         0    1.0000

>> 
\end{verbatim}
\msa{ctraj, qinterp}
\mref{R.~P. Paul, {\em Robot Manipulators: Mathematics, Programming, and Control}.
\newblock Cambridge, Massachusetts: MIT Press, 1981.}

\vfil\eject 
%%%%%%%%%%%%%%%%%%%%%%%%%%%%%%%%%%%%%%%
\function{trnorm}
\mpur{Normalize a homogeneous transformation}

\msyn{
TN = trnorm(T)
}

\mdes{Returns a normalized copy of the homogeneous transformation \var{T}.
Finite word length arithmetic can lead to homogeneous transformations in which
the rotational submatrix is not orthogonal, that is, $det({\bf R}) \neq -1$.}

\malg{Normalization is performed by orthogonalizing the
rotation submatrix $\vec{n} = \vec{o} \times \vec{a}$.}

\msa{oa2tr}
\mref{J.~Funda, ``Quaternions and homogeneous transforms in robotics,'' Master's
  thesis, University of Pennsylvania, Apr. 1988.}
\vfil\eject 

%%%%%%%%%%%%%%%%%%%%%%%%%%%%%%%%%%%%%%%
\function{twolink}
\mpur{Load kinematic and dynamic data for a simple 2-link mechanism}

\msyn{
twolink
}

\centerline{\psfig{figure=figs/twolink.eps,width=12cm}}

\mdes{Creates the \textbf{robot} object \var{tl} which describes the kinematic 
and dynamic
characteristics of a simple two-link planar manipulator.
The manipulator operates in the horizontal (XY) plane and is therefore
not influenced by gravity.

Mass is assumed to be concentrated at the joints.  All masses and
lengths are unity.
}
\msa{puma560, stanford}
\mref{Fig 3-6 of ``Robot Dynamics and Control'' by M.W. Spong and M. Vidyasagar,
1989.}
\vfil\eject 


%%%%%%%%%%%%%%%%%%%%%%%%%%%%%%%%%%%%%%%
\function{unit}
\mpur{Unitize a vector}

\msyn{
vn = unit(v)
}

\mdes{\var{unit} returns a unit vector aligned with \var{v}.}
\malg{\[
\vec{v}_n = \frac{\vec{v}}{|| \vec{v} ||}
\]}
\vfil\eject


%%%%%%%%%%%%%%%%%%%%%%%%%%%%%%%%%%%%%%%
\function{dh (legacy)}
\mpur{Matrix representation of manipulator kinematics}

\mdes{A \var{dh} matrix describes the
kinematics of a manipulator in a general way using the standard
Denavit-Hartenberg conventions.
Each row represents one link of the manipulator and the columns are assigned
according to the following table.

\begin{center}
\begin{tabular}{ccl}
Column & Symbol & \multicolumn{1}{c}{Description}\\
1 & $\alpha_i$ & link twist angle \\
2 & $A_i$ & link length\\
3 & $\theta_i$ & link rotation angle \\
4 & $D_i$ & link offset distance \\
5 & $\sigma_i$ & joint type; 0 for revolute, non-zero for prismatic \\
\end{tabular}
\end{center}

If the last column is not given, toolbox functions assume that
the manipulator is all-revolute.
For an n-axis manipulator \var{dh} is an $n \times 4$ or $n\times 5$ matrix.

The first 5 columns of a \var{dyn} matrix contain the kinematic parameters
and maybe used anywhere that a \var{dh} kinematic matrix is required --- the
dynamic data is ignored.}

Lengths $A_i$ and $D_i$ may be expressed in any unit, and this choice will flow
on to the units in which \hom s and Jacobians are represented.
All angles are in radians.

\msa{puma560,stanford,mdh}
\mref{R.~P. Paul, {\em Robot Manipulators: Mathematics, Programming, and Control}.
\newblock Cambridge, Massachusetts: MIT Press, 1981.}
\vfil\eject 



%%%%%%%%%%%%%%%%%%%%%%%%%%%%%%%%%%%%%%%
\function{dyn (legacy)}
\mpur{Matrix representation of manipulator kinematics and dynamics}

\mdes{A \var{dyn} matrix describes the
kinematics and dynamics of a manipulator in a general way using the standard
Denavit-Hartenberg conventions.
Each row represents one link of the manipulator and the columns are assigned
according to the following table.

\begin{center}
\begin{tabular}{ccl}
Column & Symbol & \multicolumn{1}{c}{Description}\\
1 & $\alpha$ & link twist angle \\
2 & A & link length \\
3 & $\theta$ & link rotation angle \\
4 & D & link offset distance \\
5 & $\sigma$ & joint type; 0 for revolute, non-zero for prismatic \\
6 & mass & mass of the link \\
7 & rx & link COG with respect to the link coordinate frame \\
8 & ry \\
9 & rz \\
10 & Ixx & elements of link inertia tensor about the link COG \\
11 & Iyy \\
12 & Izz \\
13 & Ixy \\
14 & Iyz \\
15 & Ixz \\
16 & Jm & armature inertia \\
17 & G & reduction gear ratio; joint speed/link speed \\
18 & B & viscous friction, motor referred\\
19 & Tc+ & coulomb friction (positive rotation), motor referred\\
20 & Tc- &     coulomb friction (negative rotation), motor referred\\
\end{tabular}
\end{center}

For an n-axis manipulator, \var{dyn} is an $n \times 20$ matrix.
The first 5 columns of a \var{dyn} matrix contain the kinematic parameters
and maybe used anywhere that a \var{dh} kinematic matrix is required --- the
dynamic data is ignored.
}

All angles are in radians.
The choice of all other units is up to the user, and this choice will flow
on to the units in which \hom s, Jacobians, inertias and torques
are represented.

\msa{dh}
\vfil\eject 


\cleardoublepage
\end{document}
